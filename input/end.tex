\begin{center}
\line(1,0){70}
\end{center}
\singlespacing
\quote \footnotesize Traduit de l’anglais en 2016 par Marie-Mathilde Burdeau. Corrigé, édité et imprimé en format zine en 2023 par Jérémi Robitaille-Brassard\footnote{\fontfamily{lmss}\selectfont https://jrb.nz}. Les fichiers sources du projet LaTeX sont disponibles sur Github\footnote{\fontfamily{lmss}\selectfont https://github.com/jr-b/xenofeminism-manifesto-fr}. 

\begin{center}
\line(1,0){70}
\end{center}

\footnotesize Ce zine est composé en caractère Adelphe Fructidor. Créé par Eugénie Bidaut\footnote{\fontfamily{lmss}\selectfont https://eugéniebidaut.eu}, ce caractère a pour enjeu principal de proposer plusieurs manières de pratiquer l’écriture inclusive sur du texte long, en petit corps, et sans altération du gris typographique. Son nom, qui signifie à la fois frère et sœur de manière non-genrée, est très utilisé au sein des communautés militantes queers. L’Adelphe, dans son dessin, présente des proportions classiques, héritées de la Renaissance, et un tracé proche de la calligraphie, avec une fluidité dans le ductus qui permet de produire des formes harmonieuses, y compris dans le dessin des glyphes inclusifs. Il y a 3 versions en cours de développement de l’Adelphe. L’Adelphe Germinal dans lequel le point médian est utilisé, l’Adelphe Floréal dans lequel les premières lettres des terminaisons masculines et féminines sont marquées par des signes diacritiques souscrits (accents sous les lettres), et l’Adelphe Fructidor qui combine l’usage d’une forme alternative de « e » et de ligatures. \\
