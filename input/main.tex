% Base document to build the manifesto on 4.25 in x 11 in page size
% Test on overleaf.com

\documentclass[12pt]{book}
%\usepackage[T1]{fontenc}
\usepackage{fontspec}
\usepackage[main=french]{babel}
\usepackage[verbose=silent]{microtype}
\usepackage{setspace}
\usepackage{hyperref}
\usepackage{import}
\usepackage{fancyhdr}
\usepackage[dvipsnames]{xcolor}
\usepackage{lmodern}
\usepackage{sectsty}
\allsectionsfont{\raggedleft}

% Hyphenation rules
% Can define custom hyphenation rule for specific words
%--------------------------------------
\usepackage{hyphenat}
\hyphenation{mathéma-tiques récu-pérer}
%--------------------------------------

% Typography setup
% Custom font files must be in the same directory
%--------------------------------------
\setmainfont{Adelphe-FructidorRegular.ttf}%[SizeFeatures={Size=12}]
\setlength{\emergencystretch}{2em} % https://tex.stackexchange.com/questions/111948/what-is-a-overfull-hbox-9-89561pt-too-wide
%--------------------------------------

% Page setup
% We're planning to print this on 8.5x11 inches paper, folded on the long side. We want page to be 4.25 in 
%--------------------------------------
\usepackage[paperheight=11in,paperwidth=4.15in,left=5mm,right=5mm,top=10mm,bottom=15mm]{geometry}
%--------------------------------------


% Footer & page number configuration
% 
\pagestyle{fancy}
% Clear the header and footer
\fancyhead{}
\fancyfoot{}
% Set the right side of the footer to be the page number
\fancyfoot[R]{\thepage}
\renewcommand{\headrulewidth}{0pt}


\setcounter{secnumdepth}{0}

\begin{document}



% ? Make hexa section title automatically with this: https://tex.stackexchange.com/questions/682096/change-page-numbering-in-preamble-latex-counter-modification-hexadecimal-con

% \section*{0x01} % https://www.overleaf.com/learn/latex/Questions/How_do_I_remove_the_numbers_from_section_headings%3F

\import{input/}{cover.tex}
\clearpage
\begin{quote}
\singlespacing
{\footnotesize Laboria Cuboniks\footnote{\fontfamily{lmss}\selectfont https://laboriacuboniks.net} (née en 2014) est un collectif xénoféministe, réparti dans cinq pays et sur trois continents. Elle cherche à démanteler le genre, à détruire « la famille » et à se débarrasser de la nature comme garante des positions politiques inégalitaires. Son nom est un anagramme de « Nicolas Bourbaki », pseudonyme sous lequel un groupe de mathématiciens majoritairement français a œuvré à l'affirmation de l'abstraction, de la généralité et de la rigueur en mathématiques au début du XXe siècle.}
\begin{center}
\line(1,0){70}
\end{center}
{\footnotesize Le manifeste est disponible sous licence de documentation libre GNU (\textit{GNU Free Documentation License, GFDL}), qui autorise la (re)publication, la traduction, la modification. On reproduit ici la traduction française du manifeste, avec des changements mineurs. }
\clearpage
\end{quote}

\onehalfspacing % https://texblog.org/2011/09/30/quick-note-on-line-spacing/

$body$
\newpage
\import{input/}{end.tex}

% \clearpage % https://www.overleaf.com/learn/latex/Line_breaks_and_blank_spaces#Page_breaks


\end{document}